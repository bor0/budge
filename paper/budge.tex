\documentclass{article}

\usepackage{arxiv}

\usepackage{amsmath}
\DeclareMathOperator{\sign}{sign}

% mathbb{N}
\usepackage{amsfonts}

% Code blocks
\usepackage{listings}

% Phonetic chars
\usepackage{tipa}

% Code annotation
\usepackage{listings}
\usepackage{color}

\definecolor{dkgreen}{rgb}{0,0.6,0}
\definecolor{gray}{rgb}{0.5,0.5,0.5}
\definecolor{mauve}{rgb}{0.58,0,0.82}

\lstset{frame=tb,
  language=Haskell,
  aboveskip=3mm,
  belowskip=3mm,
  showstringspaces=false,
  columns=flexible,
  basicstyle={\small\ttfamily},
  numbers=none,
  numberstyle=\tiny\color{gray},
  keywordstyle=\color{blue},
  commentstyle=\color{dkgreen},
  stringstyle=\color{mauve},
  breaklines=true,
  breakatwhitespace=true,
  frame=none,
  tabsize=3
}

\title{The Budge programming language}

\author{
  Boro Sitnikovski \\
  Skopje, North Macedonia \\
  \texttt{buritomath@gmail.com} \\
}

\newcommand\doubleplus{\mathbin{{+}\mspace{-8mu}{+}}}

\begin{document}
\maketitle

\begin{abstract}
We present a simple, esoteric programming language based on Gödel numbering and prime factorization, enhanced with explicit, scoped loops, allowing for easy program composition. We will show the syntax and semantics and then provide a few example programs and their evaluation. We will also provide a few interpreter implementations.
\end{abstract}

\keywords{Programming language, computational model, Gödel numbering, Python, Haskell, Idris}

\section{Introduction}

Budge (b\textturnv\textdyoghlig, b'dzh) is an esoteric programming language. It uses Gödel numbering\cite{b1} to represent registers and their values by relying on the Fundamental Theorem of Arithmetic\cite{b2}. For example, to represent the values $1$, $2$, and $3$ in memory, we would calculate $2^1 \cdot 3^2 \cdot 5^3$ (the first three primes 2, 3, 5 to the power of the number of the value at the corresponding register), arriving at the state $i = 2250$. We can extract $1$, $2$, and $3$ from $2250$ using prime factorization.

Budge uses similar constructs as FRACTRAN\cite{b3}. However, Budge provides a more convenient way to construct loops and uses integers rather than fractions to denote instructions. A negative integer will decrease a register's value, while a positive integer will increase a register's value. In addition, it provides an easy way to code loops by using nested parenthesis\footnote{One disadvantage of the syntax for loops is that programs can't be as easily represented in memory.}. Finally, it abstracts prime numbers in the code from the programmer.

\section{Syntax and semantics}

Where data is represented as $i \in \mathbb{N^+}$ (product of primes), the syntax of the code in Backus-Naur form\cite{b4} is:

\begin{verbatim}
<posn>  ::= "1" | "2" | ...               <negn> ::= "-1" | "-2" | ...
<stmt>  ::= <posn> | <negn> | "("<posn>","<stmts>")"
<stmts> ::= <stmt>","<stmts> | <stmt>     <code> ::= "("<stmts>")"
\end{verbatim}

Let $p(n)$ be the $n$-th prime number. Let $\sign(n) = 1$ if $n>0$ and $-1$ otherwise; this will determine whether we need to multiply or divide. Let $E(i, x)$ represent the evaluation of a program $x$ (\texttt{<code>}) for input $i$ be defined by:

$$ E(i, x) = {
\begin{cases}
E(i \cdot p(|n_0|)^{\sign(n_0)}, (n_1, \ldots, n_k)) & x = (n_0, n_1, \ldots, n_k) \land i \cdot p(|n_0|)^{\sign(n_0)} \in \mathbb{N}, \\
E(i, (n_1, \ldots, n_k)) & x = (n_0, n_1, \ldots, n_k) \land i \cdot p(|n_0|)^{\sign(n_0)} \notin \mathbb{N}, \\
E(E(i, (n'_0, \ldots, n'_k)), x) & x = ((P, n'_0, \ldots, n'_k), n_0, \ldots, n_j) \land i \cdot p(|P|)^{-1} \in \mathbb{N}, \\
E(i, (n_0, \ldots, n_j)) & x = ((P, n'_0, \ldots, n'_k), n_0, \ldots, n_j) \land i \cdot p(|P|)^{-1} \notin \mathbb{N}, \\
i & {\text{otherwise, that is, } x = ()}\end{cases}} $$

Semantically, the first case handles increasing/decreasing a value in a register $n_0$. The second case is for skipping an instruction. The third and fourth cases represent the start and end of a loop (nested parenthesis).

\section{Example programs in Budge}

\subsection{Addition of numbers (manual evaluation)}

To compute $E(2^3 \cdot 3^3, ((2, -2, 1)))$ we iterate the sequence $(-2, 1)$ until $\frac{i}{p(2)}$ is no longer an integer, that is, $\frac{i}{3}$:

\begin{enumerate}
\item Initially, $i = 2^3 \cdot 3^3 = 216$, and since $\frac{216}{3} = 72 \in \mathbb{N}$, proceed with evaluation.
\item Calculate $p(|n|)^{\sign(n)}$ for $n = -2$: $i' = p(2)^{-1} = 3^{-1}$. Since $i \cdot i' \in \mathbb{N}$, set $i$ to $i \cdot i' = 216 \cdot i' = 72$.
\item Calculate $p(|n|)^{\sign(n)}$ for $n = 1$: $i' = p(1)^{1} = 2$. Since $i \cdot i' \in \mathbb{N}$, set $i$ to $i \cdot i' = 72 \cdot i' = 144$.
\item At this point, we go back and check the condition if $\frac{144}{3} \in \mathbb{N}$ - proceed with the evaluation.
\item Calculate $p(|n|)^{\sign(n)}$ for $n = -2$: $i' = p(2)^{-1} = 3^{-1}$. Since $i \cdot i' \in \mathbb{N}$, set $i$ to $i \cdot i' = 144 \cdot i' = 48$.
\item Calculate $p(|n|)^{\sign(n)}$ for $n = 1$: $i' = p(1)^{1} = 2$. Since $i \cdot i' \in \mathbb{N}$, set $i$ to $i \cdot i' = 48 \cdot i' = 96$.
\item At this point, we go back and check the condition if $\frac{96}{3} \in \mathbb{N}$ - proceed with the evaluation.
\item Calculate $p(|n|)^{\sign(n)}$ for $n = -2$: $i' = p(2)^{-1} = 3^{-1}$. Since $i \cdot i' \in \mathbb{N}$, set $i$ to $i \cdot i' = 96 \cdot i' = 32$.
\item Calculate $p(|n|)^{\sign(n)}$ for $n = 1$: $i' = p(1)^{1} = 2$. Since $i \cdot i' \in \mathbb{N}$, set $i$ to $i \cdot i' = 32 \cdot i' = 64$.
\item Now we have that $\frac{64}{3} \notin \mathbb{N}$, so the evaluation halts.
\end{enumerate}

Thus, $i$ is now equal to $64 = 2^6$. That is, the value from the first register $p(1)$ and the value from the second register $p(2)$ were added, and then stored in the first register, $p(1)$. In general, $E(2^a \cdot 3^b, ((2, -2, 1))) = 2^n$, with $n = a + b$.

\subsection{Other arithmetic operations}

\textit{Subtraction}: $E(2^x \cdot 3^y, s) = 2^n \cdot 3^k$ where $n = x - y$ and $k=1$ if $y \leq x$, and $k=0$ otherwise.

$$s = ((1, -1, 3, 5), (2, -2, 4, 6), (3, -3, -4), (6, -5, -6), (4, -4, 1, 3), (3, (3, -3), 2), (5, -5, 1))$$

\textit{Multiplication}: $E(2^x \cdot 3^y, m) = 2^n$ where $n = x \cdot y$.

$$m = ((1, -1, (2, -2, 3, 4), (4, -4, 2)), (2, -2), (3, -3, 1))$$

\textit{Division}: $E(2^a \cdot 3^d, d) = 2^q \cdot 3^r$ where $a = qd + r$ and $0 \leq r < d$.

\begin{equation*}
\begin{split}
d = (9, (9, (2, -2, 7, 8), (8, -8, 2)) \doubleplus s \doubleplus (3, (2, -2, -3), (3, -3, 2), \\
(9, -9), (2, -2, 9), (7, -7, 2), 10), (1, -1, -2), (10, -10, 1), -1)
\end{split}
\end{equation*}

\textit{Logarithm}: $E(2^x \cdot 3^y, log) = 2^n$ where $n = \log_y(x)$.

\begin{equation*}
\begin{split}
log = ((2, -2, 12), (1, (2, -2), (14, -14), (12, -12, 2, 14), (14, -14, 12)) \doubleplus \\
d \doubleplus ((3, -3), 3, (2, -2, -3), (3, -3, 2), 11), -11, (11, -11, 1), (12, -12))
\end{split}
\end{equation*}

\subsection{Logical operators}

\textit{Logical "not"}: $E(2^x, not) = 2^n$ where $n = \neg x$.

$$not = (2,(1,-1,-2),(2,-2,1))$$

\textit{Logical "and"}: $E(2^x \cdot 3^y, and) = 2^n$ where $n = x \land y$.

$$and = ((1,(1,-1),3),(2,(2,-2),3),-3,(3,-3,1))$$

\textit{Logical "or"}: $E(2^x \cdot 3^y, or) = 2^n$ where $n = x \lor y$.

$$or = ((1,(1,-1),3),(2,(2,-2),3),(3,(3,-3),1))$$

\subsection{Representing sequences}

As an example, the sequence $(1, 2, 3)$ can be represented with the number $2^1 \cdot 3^2 \cdot 5^3 = 2250$, using Gödel numbering. We can store this number in the first register, thus $i = 2^{2250}$. To retrieve $2250$ from $2^{2250}$, we use the logarithm program. Further, to retrieve the $n$-th element of the sequence, we keep dividing by $p(n)$ until there's no remainder, and count the divisions. For example, to get the second element, $p(2) = 3$ thus $\frac{2250}{3} = 750$ is one division, and $\frac{750}{3}$ is another division (note that $\frac{250}{3}$ is not divisible) - so two divisions in total which represent the value of the index.

\subsection{Composing and interpreting programs}

As we saw with $d$ and $log$, programs can be composed by concatenating sequences: $\forall s_1, \forall s_2, E(E(i, s_1), s_2) = E(i, s_1 \doubleplus s_2)$. For example, the program $(1, 2, 2, (2, -2, 1))$ is consisted of concatenating $(1, 2, 2)$ and $((2, -2, 1))$; initialize registers with values 1 and 2 respectively, and then add 1 and 2 together, storing the result in the first register.

We show the pseudocode representation of this program by following its semantical interpretation:

\begin{verbatim}
// program 1
r1 = 1; r2 = 2;
// program 2
while (r2 > 0) { r2 -= 1; r1 += 1; } // optimized: r1 += r2; r2 = 0;
\end{verbatim}

Another way to interpret programs, to explain computation to the general audience easily, is to imagine several jars with marbles. The command $-n$ removes a marble (if any) from the jar number $n$, and $n$ adds a marble to the jar number $n$. Loops can be interpreted as a way to repeat some commands while a jar $n$ is not empty.

\section{Example interpreter implementations}

\subsection{Python}

Python\cite{b5} is a dynamically typed general-purpose programming language. For the implementation, we use lists to represent Budge programs and use recursion on them to perform the evaluation.

\begin{minipage}{0.49\textwidth}
\begin{lstlisting}[language=Python]
def evaluate(i, code):
  for el in code:
    if isinstance(el, list) and len(el) > 1:
      while i % p(abs(el[0])) == 0:
        i = evaluate(i, el[1:])
    elif sign(el) == 1:
      i *= p(el)
    else:
      new_i = p(abs(el))
      if i % new_i == 0:
        i = int(i // new_i)
  return i
\end{lstlisting}
\end{minipage}
\begin{minipage}{0.49\textwidth}
\begin{lstlisting}[language=Python]
def fast_evaluate(i, code):
  for el in code:
    if isinstance(el, list) and len(el) > 1:
      while abs(el[0]) in i and i[abs(el[0])]:
        i = fast_evaluate(i, el[1:])
    else:
      (s, el) = (sign(el), abs(el))
      if el not in i: i[el] = 0
      i[el] = max(0, i[el] + s)
  # remove zero registers
  return {x:y for x,y in i.items() if y != 0}
\end{lstlisting}
\end{minipage}

Given a function \texttt{p(n)} that returns the \texttt{n}-th prime number, the function \texttt{evaluate} closely resembles $E$. A more efficient interpreter (\texttt{fast\_evaluate}) uses a \texttt{dict} for registers and directly adjusts the values, rather than relying on number division.

An example evaluation to perform addition of two numbers:

\begin{verbatim}
>>> fast_evaluate({1:2,2:2},[[2,-2,1]])
{1: 4}
\end{verbatim}

\subsection{Haskell and Idris}

Haskell\cite{b6} is a functional and statically typed, general-purpose programming language. Idris\cite{b7} is a dependently typed programming language and allows expressing mathematical proofs. As both languages are typed, we need to provide a data type for Budge programs per the syntax.

\pagebreak

Sequences can either contain a number or a sequence, and we encode that as follows:

\begin{lstlisting}
> data Budge = BNum Int | BSeq Int [Budge]
\end{lstlisting}

Similarly to the Python implementation, given a mapping (dictionary) \texttt{Map} from numbers to numbers, we can represent \texttt{fast\_evaluate} at the value level in Haskell and at the type level in Idris:

\begin{minipage}[t]{0.49\textwidth}
\begin{lstlisting}
-- Haskell
fastEvaluate :: M.Map Int Int -> [Budge] -> M.Map Int Int
fastEvaluate i ((BNum x):xs) =
  let key   = abs x
      value = M.findWithDefault 0 (abs x) i in
      fastEvaluate (M.insert key (max 0 (value + signum x)) i) xs
fastEvaluate i l@((BSeq x' x):xs) =
  if M.findWithDefault 0 (abs x') i /= 0
  then fastEvaluate (fastEvaluate i x) l
  else fastEvaluate i xs
fastEvaluate i [] = i
\end{lstlisting}
\end{minipage}
\begin{minipage}[t]{0.49\textwidth}
\begin{lstlisting}
-- Idris
data Budge : Map -> List BudgeData -> Map -> Type where
  BId : Budge i [] i
  BPos : Budge (UpdateMap i' x (i' x + 1)) xs i''
    -> Budge i (BNum (PosNat x)::xs) (UpdateMap i' x (i' x + 1))
  BNeg : Budge (UpdateMap i' x (i' x `minus` 1)) xs i''
    -> Budge i (BNum (NegNat x)::xs) (UpdateMap i' x (i' x `minus` 1))
  BWhileT : (i x' = 0 -> Void)
    -> Budge i x i'
    -> Budge i' ((BSeq x' x)::xs) i''
    -> Budge i ((BSeq x' x)::xs) i''
  BWhileF : i x' = 0
    -> Budge i xs i'
    -> Budge i ((BSeq x' x)::xs) i'
\end{lstlisting}
\end{minipage}

The Haskell implementation uses recursion on the datatype. With it, we can run an example program:

\begin{verbatim}
> fastEvaluate (M.fromList [(1,2),(2,2)]) [BSeq 2 [BNum (-2),BNum 1]]
fromList [(1,4),(2,0)]
\end{verbatim}

Essentially, the formalization \texttt{Budge x y z} in Idris corresponds to $z = E(x, y)$. We can prove a simple fact:

\begin{lstlisting}
Addition : List BudgeData
Addition = [BSeq 2 [BNum (NegNat 2), BNum (PosNat 1)]]

-- Given state [1 -> 2, 2 -> 2], the program Addition will update the state to [1 -> 4]
EgAdd : Budge AddMap' Addition AddMap'''''
EgAdd = BWhileT
  (\case Refl impossible)
  (BNeg {i' = AddMap'} (BPos {i' = AddMap''} BId))
  (BWhileT (\case Refl impossible) (BNeg (BPos {i' = AddMap'''''} BId)) (BWhileF Refl BId))
\end{lstlisting}

\begin{thebibliography}{1}

\bibitem{b1}
Douglas Hofstadter
\newblock {\em Godel, Escher, Bach: an Eternal Golden Braid}
\newblock Basic Books, Inc., 1979.

\bibitem{b2}
Artur Korniłowicz, Piotr Rudnicki
\newblock {\em Fundamental Theorem of Arithmetic}
\newblock Formalized Mathematics 12.2, 2004.

\bibitem{b3}
John H. Conway
\newblock {\em Fractran: A simple universal programming language for arithmetic}
\newblock Open problems in Communication and Computation, 1986.

\bibitem{b4}
Daniel D. McCracken, Edwin D. Reilly
\newblock {\em{Backus-Naur Form (BNF)}}
\newblock Encyclopedia of Computer Science, 2003.

\bibitem{b5}
Guido van Rossum
\newblock Python tutorial.
\newblock {\em Technical Report CS-R9526, Centrum voor Wiskunde en Informatica (CWI), Amsterdam}, 1995.

\bibitem{b6}
Miran Lipovaca
\newblock {\em Learn You a Haskell For Great Good}
\newblock No Starch Press, 2011.

\bibitem{b7}
Boro Sitnikovski
\newblock {\em Gentle Introduction to Dependent Types with Idris}
\newblock Leanpub/Amazon KDP, 2018.

\end{thebibliography}
\end{document}
